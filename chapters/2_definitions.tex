\section{Idealism}
The word "idealism" is basically an umbrella term because, depending on the context, it could mean
something completely different. For example, this medical article published in 2006 uses the
definition of idealism as "the cherishing or pursuit of high or noble principles, purposes, or
goals” \cite{Smith2006}. Although there's no such a thing as a "wrong definition", it's certain
that said definition is not the same one used by Hegel and other self-declared idealists, for
example \cite{Beiser2009}.

As stated by Guyer and Horstmann in their entry on "The Stanford Encyclopedia of Philosophy",
there are two modern conceptions of idealism in philosophy: the one that says that something
mental is the foundation of reality (also called ontological idealism) and the one that says
that everything we can and will ever know is dependent on the mind (also called epistemological
idealism) \cite{Guyer2023}. In our case, we will be sticking with an ontological definition, as
we are going to dive deeper into the being, not the acquirement of knowledge.

Back to talking about Hegel, it's in our interest to mention his work called "Science of Logic"
(in this case, "Volume One: The Objective Logic, Book One: The Doctrine of being"), as he divides the
being into three parts of his dialectical logic: determinate being (quality), magnitude (quantity),
and measure (synthesis of both). Inspired by that division, I believe it would be wise, when talking
about idealism, to utilize the ideas of quality and quantity as a basis for the definition.

Therefore, by merging together both the ontological idealism definition given previously and also
the inspiration from Hegel, we reach to the definition that will be used on this article:

\begin{center}
    \itshape
    \parbox{0.7\textwidth}{
    "Idealism is the idea that quality is the ultimate foundation of all reality." 
    }
\end{center}

Although this definition might be seen as confusing by some, it will be explained and defended better
in the next chapter.

\section{Animism}
Animism does not suffer the same problems as idealism, for it's not really an umbrella term, but it does
suffer with another type of problem: it's a fuzzy term. The word was first coined by Sir Edward Burnett
Tylor in his work "Primitive Culture", published in 1871, calling it initially as the "doctrine of souls
and other spiritual beings" \cite{Tylor1871}.

In spite of the fact that he initially calls animism just that, he does give, on page 294, an interesting
remark, as he says:

\begin{center}
    \itshape
    \parbox{0.7\textwidth}{
    "To the theory of animism belong those endless tales which all nations tell of the presiding genii
    of nature, the spirits of the cliffs, wells, waterfalls, volcanoes, the elves and wood nymphs seen
    at times by human eyes when wandering by moonlight or assembled at their fairy festivals." 
    }
\end{center}

That observation of Tylor is intriguing, given that other authors also do characterize animism
as having those traits. Graham Harvey, in his 2005 book "Animism: respecting the living world", explores
the divergences and evolutions of the term through time \cite{Harvey2005}. Below is a table that shows
three of the most significant, in my opinion, ideas from other thinkers that were brought up by Harvey:

\begin{center}
    \begin{tabularx}{\textwidth}{||>{\raggedright\arraybackslash}X|>{\raggedleft\arraybackslash}X||} 
     \hline
     Person & Idea \\ [0.5ex] 
     \hline\hline
     David Hume & Attributing signs of human likeness, beautiful in poetry and absurd in philosophy. \\ 
     \hline
     James Frazer & Savages see and treat the world as animate, like themselves, as if the world had souls.\\
     \hline
     Irving Hallowell & Recognition of person-hood in a range of human and other-than-human persons. \\ [1ex]
     \hline
    \end{tabularx}
\end{center}

With that brief overview on the opinions related to the definition of animism, the one that will be used
on this article is the following:

\begin{center}
    \itshape
    \parbox{0.7\textwidth}{
    "Animism is the idea that everything possesses subjective experiences." 
    }
\end{center}

\section{Paganism}
The origin of the word "pagan" can be traced back to the time that the Romans officially converted to
Christianity. It was extensively used by the western roman empire as a way of mocking the countryside
folk that continued practising their ancestral religion \cite{Bowersock1999}. Even though it's considered
a pejorative term by origin, it will continue to be used on this article, but not with ill
intentions, rather because it's way more simpler to get it all bundled together in a word than to
write the same sentence over and over again.

According to Owen Davies's research written on his little introduction to paganism called "Paganism:
A Very Short Introduction", the Christians that left Europe for trade and conquest did indeed call
followers of non-Abrahamic religions as "pagans" \cite{Davies2011}. Seeing that the creators of the term
"paganism" consider it as meaning any non-Abrahamic religious tradition, therefore:

\begin{center}
    \itshape
    \parbox{0.7\textwidth}{
    "Paganism refers to any non-Abrahamic religious tradition."
    }
\end{center}

While that definition does cover many other religions, in this article, we will focus more on western
pagan religions.