\section{Quality versus quantity}
As it was defined in the last chapter, idealism puts quality as the ultimate foundation
of reality. As a result of that, and also seeing how quality and quantity were shown as
interplaying phenomena by Hegel, we can ask ourselves a certain question:

\begin{itemize}
    \item What even is quality and quantity?
\end{itemize}

Quality, in this context, might differ a little from the usual and casual definition.
In day to day conversation, quality might be used as an indicative of how "good" or
"bad" something is in said context. However, the definition that will be used here is
more related to the term commonly used in philosophy called "qualia".

As the first to introduce the term in it's modern sense, Clarence Irving Lewis defined
the word "qualia" as the immediate phenomenal qualities of experience, that is, the
redness of red, the fluffiness of a cat, the loudness of a firearm \cite{Lewis1956}.
With the word quality, that's what's meant here, granted that the casual way also
can be applied as a quality of "goodness" and "badness", as in feeling the good and
feeling the bad of said thing.

Now that the idea of quality is grasped, it's easy to define quantity. In a nutshell,
just as quality is related to qualia, quantity is related to another term named "quanta".
Unlike qualia, quanta doesn't trace that well to any specific first "creator" of the
modern definition, however it was used by lots of physicists, including famous figures
like Max Planck, for example \cite{Planck1901}. Quanta, as the name might or might not
suggest to you, is related to quantities, just like in the phrase "I have two oranges",
where "orange" is the unit that is being quantified and "two" is the quantity.

Last but not least, there are some final questions that might have been asked, they are
the following ones:

\begin{itemize}
    \item If quality is the foundation of reality to idealists, what would someone that
    follows quantity as the foundation be called?
    \item What about the idea that both quality and quantity coexist as foundations?
    \item Why not a third thing?
\end{itemize}

In summary, while idealists view quality as the foundation of reality, materialists
consider quantity to be its basis. When both perspectives coexist, this framework is
known as dualism. Alternatively, the notion of a third, neutral foundation, one that
relies on neither quality nor quantity, is called neutral monism. These concepts will
be explored in greater detail in the following sections of this chapter.

\section{Problems of materialism}
When we talk about materialism as placing quantities at the foundation of reality,
one might ask:

\begin{itemize}
    \item Quantities of what?
\end{itemize}

Materialism typically reduces everything to measurable, physical quantities, such as mass,
energy, particles, and forces, and constructs units of measurement based on them, such as
kilograms, joules, and meters. But what, fundamentally, is a kilogram? Or what is
a joule? While organizations like the International Bureau of Weights and
Measures (BIPM) provide definitions for these units, there is something deeply
problematic about this framework: it doesn't really state what is the quantity for.

According to the General Conference on Weights and Measures, the supreme authority of
BIPM, the unit of measurement "second" is defined as the following:

\begin{center}
    \itshape
    \parbox{0.7\textwidth}{
    "The second, symbol s, is the SI unit of time. It is defined by taking the fixed
    numerical value of the caesium frequency $\Delta\nu_{\text{Cs}}$, the unperturbed
    ground-state hyperfine transition frequency of the caesium-133 atom, to be 9.192.631.770
    when expressed in the unit Hz, which is equal to $s^{-1}$."\\
    \normalfont - BIPM \cite{BIPMSecond}
    }
\end{center}

Notice anything unusual? They use quantities to define the unit of quantity. Let's give
another example:

\begin{center}
    \itshape
    \parbox{0.7\textwidth}{
    "The metre, symbol m, is the SI unit of length. It is defined by taking the fixed
    numerical value of the speed of light in vacuum c to be 299.792.458 when expressed in
    the unit m $s^{-1}$, where the second is defined in terms of the caesium frequency
    $\Delta\nu_{\text{Cs}}$."\\
    \normalfont - BIPM \cite{BIPMMetre}
    }
\end{center}

The seven standard units defined by the BIPM (second, metre, kilogram, ampere, kelvin, mole,
candela) all follow the same pattern: they utilize the definition of other standard units
and/or they use quantities to define said unit, not really specifying what it's really
quantifying.

It is indeed possible to write an entire book about that specific critique and it's also not
so hard to give a good response against it, but a more important, and harder to counter,
critique is related to qualities directly:

\begin{itemize}
    \item How can, in a world of only quantities, a quality exist?
\end{itemize}

This question is what brings into light the infamous hard problem of
consciousness.

The hard problem of consciousness, as first formulated by David Chalmers in his article
called "Facing up to the problem of consciousness", is explained as, in his own words,
"the problem of experience":

\begin{center}
    \itshape
    \parbox{0.7\textwidth}{
    "It is undeniable that some organisms are subjects of experience. But the question of how
    it is that these systems are subjects of experience is perplexing. Why is it that when our
    cognitive systems engage in visual and auditory information-processing, we have visual or
    auditory experience: the quality of deep blue, the sensation of middle C? How can we explain
    why there is something it is like to entertain a mental image, or to experience an emotion? It
    is widely agreed that experience arises from a physical basis, but we have no good
    explanation of why and how it so arises. Why should physical processing give rise to a rich
    inner life at all? It seems objectively unreasonable that it should, and yet it does."\\
    \normalfont - David Chalmers \cite{Chalmers1995}
    }
\end{center}

The materialist, trying to defend his position when coming into contact with said contradiction,
usually chooses one of these positions: eliminativism, reductionism, representationalism, or
illusionism. Eliminativists, like Paul Churchland, would say that qualities (or qualia) does not
really exist \cite{Churchland1981}. Reductionists, like Daniel Dennett, would prefer saying that
it can all be reduced down to how your brain functions \cite{Dennett1991}. Representationalists,
like Michael Tye, defend the idea that qualities are just representational states of the brain,
nothing more and nothing else \cite{Tye1997}. Lastly, illusionists think the qualities are just
illusions, like how Keith Frankish thinks \cite{Frankish2016}.

In contrast to their self proclaimed differences in thought, they all can be summarized as
just in denial, considering that neither of them really bridge the gap that is left between
their quantitative world and the qualities we perceive. One example given by the reductionist
Daniel Dennett is that qualities are just the "desktop interface of a computer", foolishly
forgetting that the desktop interface just is a desktop interface because we experience them
with our own eyes, therefore being an useless example. At last, it's important to reiterate
that there's no knowledge without experiences, and the mistakes of materialism just
reinforce that.

\section{Dualism and neutral monism}
Just as explained in the first section of this chapter, there are two other views that need
to be explored: dualism and neutral monism. Recapitulating, dualism is the view that both
quality and quantity are the ultimate foundations of reality and neutral monism is the view
that there's another third thing that is the ultimate foundation.

Starting with the critique of the latter, one of the main problems I personally see with
neutral monism is the lack of alternatives:

\begin{itemize}
    \item If it's not quality nor quantity, what else is there to be?
\end{itemize}

Although it sounds like a silly critique, it does make sense and it is
just as valid, considering the neutral monists are the ones that need to make it lack
vagueness. Some neutral monists, like Ernst Mach, are just closeted idealists, considering
that they see the "neutral one" as being qualitative or, in the case of Mach, "sensations"
\cite{Mach1914}.

Referring to dualism, the most lazy, in my opinion, critique of it is the one done by
Bertrand Russell in his book "The Analysis of Mind", which is basically just the application
of Occam's razor and it's just weak, considering the razor doesn't apply everywhere
\cite{Russell1921}. A rhetorical question that I find interesting to ask yourself when
talking about dualism is the following one:

\begin{itemize}
    \item If everything had the quality of yellowness and only that, would quantity even exist?
\end{itemize}

Another alternative would be to localize it instead of making it apply globally to the entire
reality:

\begin{itemize}
    \item If there's somewhere in reality that only the quality of yellowness exist, where's the
    quantity there?
\end{itemize}

Considering that dualists see quantity as being foundational to reality just as quality is, there
should be quantity present in those cases. Another remark that arises by using those rhetorical
questions is in relation to the uniqueness of a quality, as if it's not possible to isolate a
quality, then it's not really a unique quality.